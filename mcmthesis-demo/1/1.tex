\subsection{Problem Background}

As one of the oldest ecosystems on earth, forest ecosystems have played an integral role in the evolution process of the planet's ecology. They have potent ecological functions such as maintaining water and soil, protecting biodiversity and regulating climate, which reflect great ecological values. Among them, the function of carbon sequestration is the most prominent. Carbon sequestration refers to a process where forests sequester carbon dioxide out of the atmosphere, which plays an essential part in the biosphere cycle. However, the development of the forest is twisting. Due to the lack of environmental awareness and effective management, the earth's forests have been destroyed on a large scale over the past few centuries. Ecology and economy seem to be on opposite sides when both are considered. Therefore, economic and ecological forest management plans are urgently needed.

%And as human civilization's need for development grows stronger, so does the demand for land resources. 

%As a result, the earth's forests have been destroyed on a large scale over the past few centuries, and the lack of effective conservation and management measures has led to an increasingly serious shrinkage of forests, resulting in a rapid deterioration of the ecological environment.One of the legitimate causes of these problems is the lack of a deeper understanding of the ecological value of land use.

%In recent decades, the increasing consequences of environmental degradation have led to a greater awareness of the value of forest ecosystems' ecological functions. Through the establishment of nature reserves, afforestation, and effective forestry techniques, the forests on earth are gradually being restored with the help of these techniques, allowing people to obtain forestry resources from them without damaging the ecosystem, and realizing the economic value to the forest. However, there is still a need to consider how to effectively manage the forest resources in order to optimise the ecological and economic value of the forests.




